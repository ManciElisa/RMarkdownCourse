% Options for packages loaded elsewhere
\PassOptionsToPackage{unicode}{hyperref}
\PassOptionsToPackage{hyphens}{url}
%
\documentclass[
  ignorenonframetext,
]{beamer}
\usepackage{pgfpages}
\setbeamertemplate{caption}[numbered]
\setbeamertemplate{caption label separator}{: }
\setbeamercolor{caption name}{fg=normal text.fg}
\beamertemplatenavigationsymbolsempty
% Prevent slide breaks in the middle of a paragraph
\widowpenalties 1 10000
\raggedbottom
\setbeamertemplate{part page}{
  \centering
  \begin{beamercolorbox}[sep=16pt,center]{part title}
    \usebeamerfont{part title}\insertpart\par
  \end{beamercolorbox}
}
\setbeamertemplate{section page}{
  \centering
  \begin{beamercolorbox}[sep=12pt,center]{part title}
    \usebeamerfont{section title}\insertsection\par
  \end{beamercolorbox}
}
\setbeamertemplate{subsection page}{
  \centering
  \begin{beamercolorbox}[sep=8pt,center]{part title}
    \usebeamerfont{subsection title}\insertsubsection\par
  \end{beamercolorbox}
}
\AtBeginPart{
  \frame{\partpage}
}
\AtBeginSection{
  \ifbibliography
  \else
    \frame{\sectionpage}
  \fi
}
\AtBeginSubsection{
  \frame{\subsectionpage}
}
\usepackage{amsmath,amssymb}
\usepackage{lmodern}
\usepackage{iftex}
\ifPDFTeX
  \usepackage[T1]{fontenc}
  \usepackage[utf8]{inputenc}
  \usepackage{textcomp} % provide euro and other symbols
\else % if luatex or xetex
  \usepackage{unicode-math}
  \defaultfontfeatures{Scale=MatchLowercase}
  \defaultfontfeatures[\rmfamily]{Ligatures=TeX,Scale=1}
\fi
% Use upquote if available, for straight quotes in verbatim environments
\IfFileExists{upquote.sty}{\usepackage{upquote}}{}
\IfFileExists{microtype.sty}{% use microtype if available
  \usepackage[]{microtype}
  \UseMicrotypeSet[protrusion]{basicmath} % disable protrusion for tt fonts
}{}
\makeatletter
\@ifundefined{KOMAClassName}{% if non-KOMA class
  \IfFileExists{parskip.sty}{%
    \usepackage{parskip}
  }{% else
    \setlength{\parindent}{0pt}
    \setlength{\parskip}{6pt plus 2pt minus 1pt}}
}{% if KOMA class
  \KOMAoptions{parskip=half}}
\makeatother
\usepackage{xcolor}
\newif\ifbibliography
\usepackage{color}
\usepackage{fancyvrb}
\newcommand{\VerbBar}{|}
\newcommand{\VERB}{\Verb[commandchars=\\\{\}]}
\DefineVerbatimEnvironment{Highlighting}{Verbatim}{commandchars=\\\{\}}
% Add ',fontsize=\small' for more characters per line
\usepackage{framed}
\definecolor{shadecolor}{RGB}{248,248,248}
\newenvironment{Shaded}{\begin{snugshade}}{\end{snugshade}}
\newcommand{\AlertTok}[1]{\textcolor[rgb]{0.94,0.16,0.16}{#1}}
\newcommand{\AnnotationTok}[1]{\textcolor[rgb]{0.56,0.35,0.01}{\textbf{\textit{#1}}}}
\newcommand{\AttributeTok}[1]{\textcolor[rgb]{0.77,0.63,0.00}{#1}}
\newcommand{\BaseNTok}[1]{\textcolor[rgb]{0.00,0.00,0.81}{#1}}
\newcommand{\BuiltInTok}[1]{#1}
\newcommand{\CharTok}[1]{\textcolor[rgb]{0.31,0.60,0.02}{#1}}
\newcommand{\CommentTok}[1]{\textcolor[rgb]{0.56,0.35,0.01}{\textit{#1}}}
\newcommand{\CommentVarTok}[1]{\textcolor[rgb]{0.56,0.35,0.01}{\textbf{\textit{#1}}}}
\newcommand{\ConstantTok}[1]{\textcolor[rgb]{0.00,0.00,0.00}{#1}}
\newcommand{\ControlFlowTok}[1]{\textcolor[rgb]{0.13,0.29,0.53}{\textbf{#1}}}
\newcommand{\DataTypeTok}[1]{\textcolor[rgb]{0.13,0.29,0.53}{#1}}
\newcommand{\DecValTok}[1]{\textcolor[rgb]{0.00,0.00,0.81}{#1}}
\newcommand{\DocumentationTok}[1]{\textcolor[rgb]{0.56,0.35,0.01}{\textbf{\textit{#1}}}}
\newcommand{\ErrorTok}[1]{\textcolor[rgb]{0.64,0.00,0.00}{\textbf{#1}}}
\newcommand{\ExtensionTok}[1]{#1}
\newcommand{\FloatTok}[1]{\textcolor[rgb]{0.00,0.00,0.81}{#1}}
\newcommand{\FunctionTok}[1]{\textcolor[rgb]{0.00,0.00,0.00}{#1}}
\newcommand{\ImportTok}[1]{#1}
\newcommand{\InformationTok}[1]{\textcolor[rgb]{0.56,0.35,0.01}{\textbf{\textit{#1}}}}
\newcommand{\KeywordTok}[1]{\textcolor[rgb]{0.13,0.29,0.53}{\textbf{#1}}}
\newcommand{\NormalTok}[1]{#1}
\newcommand{\OperatorTok}[1]{\textcolor[rgb]{0.81,0.36,0.00}{\textbf{#1}}}
\newcommand{\OtherTok}[1]{\textcolor[rgb]{0.56,0.35,0.01}{#1}}
\newcommand{\PreprocessorTok}[1]{\textcolor[rgb]{0.56,0.35,0.01}{\textit{#1}}}
\newcommand{\RegionMarkerTok}[1]{#1}
\newcommand{\SpecialCharTok}[1]{\textcolor[rgb]{0.00,0.00,0.00}{#1}}
\newcommand{\SpecialStringTok}[1]{\textcolor[rgb]{0.31,0.60,0.02}{#1}}
\newcommand{\StringTok}[1]{\textcolor[rgb]{0.31,0.60,0.02}{#1}}
\newcommand{\VariableTok}[1]{\textcolor[rgb]{0.00,0.00,0.00}{#1}}
\newcommand{\VerbatimStringTok}[1]{\textcolor[rgb]{0.31,0.60,0.02}{#1}}
\newcommand{\WarningTok}[1]{\textcolor[rgb]{0.56,0.35,0.01}{\textbf{\textit{#1}}}}
\usepackage{graphicx}
\makeatletter
\def\maxwidth{\ifdim\Gin@nat@width>\linewidth\linewidth\else\Gin@nat@width\fi}
\def\maxheight{\ifdim\Gin@nat@height>\textheight\textheight\else\Gin@nat@height\fi}
\makeatother
% Scale images if necessary, so that they will not overflow the page
% margins by default, and it is still possible to overwrite the defaults
% using explicit options in \includegraphics[width, height, ...]{}
\setkeys{Gin}{width=\maxwidth,height=\maxheight,keepaspectratio}
% Set default figure placement to htbp
\makeatletter
\def\fps@figure{htbp}
\makeatother
\setlength{\emergencystretch}{3em} % prevent overfull lines
\providecommand{\tightlist}{%
  \setlength{\itemsep}{0pt}\setlength{\parskip}{0pt}}
\setcounter{secnumdepth}{-\maxdimen} % remove section numbering
\ifLuaTeX
  \usepackage{selnolig}  % disable illegal ligatures
\fi
\IfFileExists{bookmark.sty}{\usepackage{bookmark}}{\usepackage{hyperref}}
\IfFileExists{xurl.sty}{\usepackage{xurl}}{} % add URL line breaks if available
\urlstyle{same} % disable monospaced font for URLs
\hypersetup{
  pdftitle={Presentation\_ioslides},
  pdfauthor={Elisa Mancinelli},
  hidelinks,
  pdfcreator={LaTeX via pandoc}}

\title{Presentation\_ioslides}
\author{Elisa Mancinelli}
\date{08-03-2023}

\begin{document}
\frame{\titlepage}

\begin{frame}[fragile]
\begin{Shaded}
\begin{Highlighting}[]
\NormalTok{pre \{ }
  \KeywordTok{max{-}height}\NormalTok{: }\DecValTok{700}\DataTypeTok{px}\OperatorTok{;}
  \KeywordTok{overflow{-}y}\NormalTok{: }\BuiltInTok{auto}\OperatorTok{;}
\NormalTok{\}}

\NormalTok{pre}\ExtensionTok{[class]}\NormalTok{ \{}
  \KeywordTok{max{-}height}\NormalTok{: }\DecValTok{500}\DataTypeTok{px}\OperatorTok{;}
\NormalTok{\}}

\FunctionTok{.scroll{-}100}\NormalTok{ \{}
  \KeywordTok{max{-}height}\NormalTok{: }\DecValTok{500}\DataTypeTok{px}\OperatorTok{;}
  \KeywordTok{overflow{-}y}\NormalTok{: }\BuiltInTok{auto}\OperatorTok{;}
  \KeywordTok{background{-}color}\NormalTok{: }\BuiltInTok{inherit}\OperatorTok{;}
\NormalTok{\}}
\end{Highlighting}
\end{Shaded}

```
\end{frame}

\hypertarget{esercizi---fare-le-slide}{%
\section{ESERCIZI - FARE LE SLIDE}\label{esercizi---fare-le-slide}}

\begin{frame}{Slide 1 testo colorato e cambio font}
\protect\hypertarget{slide-1-testo-colorato-e-cambio-font}{}
{ bla bla bla }

bla bla bla
\end{frame}

\begin{frame}{Slide 2 immagine con Markdown}
\protect\hypertarget{slide-2-immagine-con-markdown}{}
\begin{figure}
\centering
\includegraphics{image/elefante.jpg}
\caption{{Bla bla}}
\end{figure}
\end{frame}

\begin{frame}{Slide 3 immagine con RMarkdown}
\protect\hypertarget{slide-3-immagine-con-rmarkdown}{}
\begin{figure}

{\centering \includegraphics[width=0.6\linewidth]{image/daydreaming5final} 

}

\caption{Mega-testa colorata}\label{fig:unnamed-chunk-1}
\end{figure}
\end{frame}

\begin{frame}{Slide 4 immagine nel testo}
\protect\hypertarget{slide-4-immagine-nel-testo}{}
In questa slide mettiamo un {\emph{immagine}} dentro il testo

\begin{itemize}
\tightlist
\item
  bla
\item
  bla bla
\item
  bal bla bla
\end{itemize}
\end{frame}

\begin{frame}{Slide 5 - contenuti incrementali}
\protect\hypertarget{slide-5---contenuti-incrementali}{}
Nomi

Cose

Città

gne gne
\end{frame}

\begin{frame}{Slide 6 - più colonne}
\protect\hypertarget{slide-6---piuxf9-colonne}{}
trallalero

trallallà
\end{frame}

\begin{frame}{slide 7 - colonne - metodo con div}
\protect\hypertarget{slide-7---colonne---metodo-con-div}{}
{Trallalero}

{trallallà}

Un elefante
\end{frame}

\begin{frame}{slide 8 - chunk di codice}
\protect\hypertarget{slide-8---chunk-di-codice}{}
ciò che si è detto nelle lezioni precedenti sui chunk vale anche qui,
per le slides

\begin{itemize}
\tightlist
\item
  ridurre il font del codice

  \begin{itemize}
  \tightlist
  \item
    mettere myClass, presente sopra il primo chunk
  \end{itemize}
\end{itemize}
\end{frame}

\end{document}
